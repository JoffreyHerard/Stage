\documentclass[french]{article}
\usepackage{amssymb, amsmath, mathtools} %pour les mathématiques
\usepackage{fontspec}
\usepackage{xunicode}
\usepackage[a4paper]{geometry}
\usepackage{babel}
\usepackage{hyperref}
\usepackage{pifont}
\usepackage{tikz}
\usepackage{listings}
\usepackage{minted}
\lstset{basicstyle=\ttfamily,
  showstringspaces=false,
  commentstyle=\color{red},
  keywordstyle=\color{blue}
}
\usetikzlibrary{arrows.meta,shapes,positioning,shadows,trees}

\tikzset{
    basic/.style  = {draw, text width=2cm, drop shadow, font=\sffamily, rectangle},
    root/.style   = {basic, rounded corners=2pt, thin, align=center,
                     fill=green!30},
    onode/.style = {basic, thin, align=center, fill=green!60,text width=3cm,},
    tnode/.style = {basic, thin, align=left, fill=pink!60, text width=6.5em},
    edge from parent/.style={->, >={latex}, draw=black, edge from parent fork right}
}

\newcommand{\cmark}{\ding{51}}%
\newcommand{\xmark}{\ding{55}}%

\newtheorem{post}{Postulat}
\newtheorem{mydef}{Définition}

\begin{document}
\title{To Do List}
\author{Joffrey Hérard}
\date{\today} 
\maketitle
\section{Today}
\begin{enumerate}
\item Mise en Place d'un Planning
\item Définitions des termes du sujet
\item Définitions des cas d'utilisations.
\item Recherche Bibliographique 
\item Recherche sur les technologies existantes
\item Choix des Hyperviseurs et Conteneurs
\item Choix de ce qui va être testé . 
\item Mise en place machines physiques et préparation des tests .
\item Lancement des tests
\item Analyse Résultats
\item Rédaction
\end{enumerate}
\newpage
\section{Archives}
\paragraph{Listes des choses à faire Mis a jour 4/04/2017}
\begin{enumerate}
	\item Définitions des cas d'utilisations. 
	\item Choix des Hyperviseurs de Type 1 ? Type 2 ? Conteneurs?
	\item Choix cible des Tests.
	\item Définitions des différents scenarios associé au différent schéma ..
	\item Préparations des différents scripts associés aux scenarios pour les différent hyperviseur, conteneurs 
	\item Remise à zero de la machine hôte.
	\item Mise en place d'une << sonde >> pour effectué les relevés de sorte à être des plus neutre dans les enregistrements.
	\item Mise en place des Tests 
		\begin{enumerate}
			\item Préparations de la machine physique et des machines virtuelles. 
			\item Lancement des tests.
			\item Surveillance régulière du bon déroulement des différents test.
			\item Récupérations des données . 
			\item Analyse des résultats sur la cohérence .
			\begin{enumerate}
				\item Établir des liens entre certain paramètres $\rightarrow $ les confirmers par d'autre test ? 
				\item Analyse sur les différents hyperviseur/conteneurs sur les mêmes scenarios 
			\end{enumerate}
			\item Représentations des résultats pour chaque batterie de tests. (Courbes) et les interprétations.
			\item Retour à l’étape (a) si nécessaire sinon étapes suivantes.
		\end{enumerate}
	\item Écriture du rapport au fur et mesure.	
			
\end{enumerate}

\end{document}
