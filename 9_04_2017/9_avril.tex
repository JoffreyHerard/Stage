\documentclass[french]{article}
\usepackage{amssymb, amsmath, mathtools} %pour les mathématiques
\usepackage{fontspec}
\usepackage{xunicode}
\usepackage[a4paper]{geometry}
\usepackage{babel}
\usepackage{hyperref}
\usepackage{pifont}
\usepackage{minted}
\usepackage{tikz}
\usepackage{listings}
\lstset{basicstyle=\ttfamily,
  showstringspaces=false,
  commentstyle=\color{red},
  keywordstyle=\color{blue}
}
\usetikzlibrary{arrows.meta,shapes,positioning,shadows,trees}

\tikzset{
    basic/.style  = {draw, text width=2cm, drop shadow, font=\sffamily, rectangle},
    root/.style   = {basic, rounded corners=2pt, thin, align=center,
                     fill=green!30},
    onode/.style = {basic, thin, align=center, fill=green!60,text width=3cm,},
    tnode/.style = {basic, thin, align=left, fill=pink!60, text width=6.5em},
    edge from parent/.style={->, >={latex}, draw=black, edge from parent fork right}
}

\newcommand{\cmark}{\ding{51}}%
\newcommand{\xmark}{\ding{55}}%

\newtheorem{post}{Postulat}
\newtheorem{mydef}{Définition}

\begin{document}
\title{Résumé Journalier}
\author{Joffrey Hérard}
\date{9 avril 2017} 

\maketitle

\section{Postulats}

\section{Scripts d'automatisation des taches}
RAPPEL : 
Lancement d'un test après configuration du batch-setup :

\begin{minted}{bash}
phoronix-test-suite batch-benchmark chess
\end{minted}

Trouver le nom du dernier test effectuer par Phoronix sur une debian : 
\begin{minted}{bash}
ls -t ~/.phoronix-test-suite/test-results | sed -n '1p'
\end{minted}



\begin{lstlisting}[language=bash,caption={install.sh},frame=single]
#!/bin/bash

sudo apt-get install phoronix-test-suite
\end{lstlisting}


\begin{lstlisting}[language=bash,caption={launchJob.sh},frame=single]
#!/bin/bash
phoronix-test-suite batch-benchmark $1
\end{lstlisting}
\newpage
\begin{lstlisting}[language=bash,caption={packResult.sh},frame=single]
#!/bin/bash

chemin=".phoronix-test-suite/test-results"
dernier_test=$(ls -t ~/.phoronix-test-suite/test-results|sed -n '1p' )
echo dossier = "$dernier_test"
echo chemin = "$chemin"

new_path=$chemin/
cd 
home=`pwd`
echo home = $home
cd $new_path

zip -r RT.zip $dernier_test
mv RT.zip $home

#Envoyer a l'hote
\end{lstlisting}

Le dernier élément a concevoir, reste évidemment la mise en place de l'envoi des résultats il y a plusieurs manière de faire, la première serait de mettre en place un réseaux avec L’Hôte, puis de lui envoyer de quelque manière que ce soit : \newline
SCP, SFTP, FTP .
A la rigueur on peut donc mettre en place pour automatiser tout cela, un serveur externe à L’Hôte pour donner l'intégralité des performances aux machines virtuelles et donc empêcher un paramètre supplémentaire de corrompre d’éventuel résultats. Dans ce cas chaque machines virtuelles devras-être connecté a Internet, par conséquent on doit mettre en place un serveur joignable depuis les machines invités.


\paragraph{Envoie des JOB a tout les invités}  
\subparagraph{SaltStack}
Pour l'automatisation des taches il est assez aisé de pouvoir envoyer des commandes d’exécution à chaque machines virtuelle "invités" ou encore esclaves. La machine hôte serait doté d'un salt master, et l'ensemble des invités serais des salt-minions. 
 
\paragraph{Réception des JOB à tout les invités}  
\subparagraph{SCP}
Mise en place de clé privé/publique pour automatisation des échanges. Mis en place d'un serveur SSH sur l'hote $\rightarrow$ Ressource supplémentaires de Saltstack ; ou externe à la machine.
\subparagraph{FTP}
Mise en place d'un serveur FTP externe $\rightarrow$ Ressource supplémentaire avec Saltstack 
\subparagraph{SFTP}
Mise en place d'un serveur SFTP externe $\rightarrow$ Ressource supplémentaire avec Saltstack 
\subparagraph{SaltStack}

Il est possible avec les fonctions 
salt.modules.cp.get\_file(path, dest, saltenv='base', makedirs=False, template=None, gzip=None, **kwargs)
\begin{lstlisting}[language=bash,caption={Salt copy },frame=single]
salt '*' cp.get_file salt://path/to/file /minion/dest
\end{lstlisting}

\section{Test en local}

Aujourd'hui, test sur l'hyperviseur VirtualBox, de simuler une machine hôte debian 8.7.1 avec des invité LXC avec SaltStack pour la gestions des différents script.
\newpage
\begin{thebibliography}{9}

   \bibitem{Site de SaltStack}
        Site officiel de SaltStack
          \url{https://saltstack.com/}.

   \bibitem{Documentation de SaltStack}
         Documentation de SaltStack.
          \url{https://docs.saltstack.com/en/latest/}.
\end{thebibliography}

\end{document}