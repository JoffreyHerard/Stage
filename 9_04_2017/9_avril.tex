\documentclass[french]{article}
\usepackage{amssymb, amsmath, mathtools} %pour les mathématiques
\usepackage{fontspec}
\usepackage{xunicode}
\usepackage[a4paper]{geometry}
\usepackage{babel}
\usepackage{hyperref}
\usepackage{pifont}
\usepackage{minted}
\usepackage{tikz}
\usetikzlibrary{arrows.meta,shapes,positioning,shadows,trees}

\tikzset{
    basic/.style  = {draw, text width=2cm, drop shadow, font=\sffamily, rectangle},
    root/.style   = {basic, rounded corners=2pt, thin, align=center,
                     fill=green!30},
    onode/.style = {basic, thin, align=center, fill=green!60,text width=3cm,},
    tnode/.style = {basic, thin, align=left, fill=pink!60, text width=6.5em},
    edge from parent/.style={->, >={latex}, draw=black, edge from parent fork right}
}

\newcommand{\cmark}{\ding{51}}%
\newcommand{\xmark}{\ding{55}}%

\newtheorem{post}{Postulat}
\newtheorem{mydef}{Définition}

\begin{document}
\title{Résumé Journalier}
\author{Joffrey Hérard}
\date{8 avril 2017} 

\maketitle

\section{Postulats}

\begin{post}
Chaque machines virtuelles qu'elle soit engendre d'un Hyperviseur ou d'un service de Conteneurisation seras une machine debian.  
\end{post}
\section{Scripts d'automatisation des taches}
RAPPEL : 
Lancement d'un test après configuration du batch-setup :

\begin{minted}{bash}
phoronix-test-suite batch-benchmark chess
\end{minted}

Trouver le nom du dernier test effectuer par Phoronix sur une debian : 
\begin{minted}{bash}
ls -t ~/.phoronix-test-suite/test-results | sed -n '1p'
\end{minted}
\newpage
\begin{thebibliography}{9}
\end{thebibliography}

\end{document}