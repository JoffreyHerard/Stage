\chapter{Présentation du projet}

Au travers de cette présentation il sera présenté l'ensemble du sujet, ainsi que les problématiques qu'il soulève, une définition précise des différents termes employés, l'explication des choix qui ont été fait sur certaines technologies et enfin l'ensemble des solutions émises pour résoudre les problématiques levées.
\section{Sujet}

Ce stage ayant pour sujet la mise en place d'une solution d'évaluation des performances de solutions de virtualisation basée sur une expérience utilisateur, il a tout d'abord été nécessaire de définir plusieurs termes afin de cerner ce dont-il était vraiment question. Ainsi, j'ai d'abord déterminé les termes qu'il était intéressant de développer, tels que : les solutions de virtualisation, l'évaluation de ces solutions et l'expérience utilisateur. Pour commencer, portons notre attention sur les solutions de virtualisation afin d'illustrer de quoi il s'agit. 


\begin{mydef}
Solutions de virtualisation : La virtualisation consiste à faire fonctionner un ou plusieurs systèmes d'exploitation / applications comme un simple logiciel, sur un ou plusieurs ordinateurs. Actuellement séparé en deux grands domaines, l’hyper-vision (eux même divisés en deux types : Type 1 et type 2) et la conteneurisation.
\end{mydef}

Après avoir défini en détails les termes des solutions de virtualisation il est nécessaire d'expliquer le processus d'évaluation des performances. 

\newpage
\section{Objectifs}
\begin{enumerate}
	\item Savoir évaluer les différentes solutions de virtualisation. 	
	\item Effectuer une démarche scientifique cohérente.
	\item Effectuer un travail collaboratif avec des personnes d’expériences sur le sujet, ce qui malgré tout est plus ou moins inédit pour moi (J’entends que la plupart des travaux réalisés ces dernières années ont été fait avec des gens de même expérience que moi).
	\item Établir des statistiques, sur les résultats qui seront obtenus. En ressortir des résultats des variables qui sortent du lot.
	\item Établir un tableau récapitulatif afin de pouvoir résumer toutes les données obtenues.
\end{enumerate}

\section{Problématique soulevée}

\begin{center} 
Comment mettre en oeuvre des scénarios afin d'évaluer les différentes solutions de virtualisation et de conteneurisation du point de vue utilisateurs ? Tout cela de manière efficace et juste sans interférer dans les résultats.
\end{center}

\section{Hypothèse de solution}

%Quoi :
Il existe un ensemble de logiciels type pour le provisionning de machine virtuelle ainsi que des outils permettant de réaliser des benchmark neutres dans leurs prises de ressources.

