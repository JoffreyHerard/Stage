\renewcommand{\abstractnamefont}{\normalfont\Large\bfseries}
%\renewcommand{\abstracttextfont}{\normalfont\Huge}

\begin{abstract}
\hskip7mm

\begin{spacing}{1.3}
L'objet de notre étude a porté sur l'ensemble des solutions de virtualisation disponibles et possibles dans le cadre de la machine support mise à disposition. Il semble essentiel de rappeler que ces solutions sont variées dans leur fondamentalisme, leurs principes ou bien encore dans leurs domaines traditionnels d'utilisation. Afin de mettre en place une solution d'évaluation des solutions de virtualisation du point de vue des utilisateurs nous avons défini des scénarios de tests et des domaines d'études spécifiques de test sur chaque composant du système qui pourrait être utilisés par un ensemble d'utilisateurs. Pour cela, il a fallut définir ce qui allait être étudié sur chaque composant mais aussi comment mettre en oeuvre ces tests sur l'ensemble de ces machines. Je pus alors me poser un certain nombre de questions : Quels outils existent ? Comment faire la mise en place de ces outils ? Comment faire l'analyse de l'ensemble des résultats ? Comment faire l'établissement des liens et d'une échelle de performances ? Comment faire pour délimiter les meilleurs domaines de chaque solutions de virtualisation ?
\end{spacing}
\end{abstract}
