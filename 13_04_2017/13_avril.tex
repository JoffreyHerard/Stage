\documentclass[french]{article}
\usepackage{amssymb, amsmath, mathtools} %pour les mathématiques
\usepackage{fontspec}
\usepackage{xunicode}
\usepackage[a4paper]{geometry}
\usepackage{babel}
\usepackage{hyperref}
\usepackage{pifont}
\usepackage{tikz}
\usepackage{listings}
\usepackage{minted}
\lstset{basicstyle=\ttfamily,
  showstringspaces=false,
  commentstyle=\color{red},
  keywordstyle=\color{blue}
}
\usetikzlibrary{arrows.meta,shapes,positioning,shadows,trees}

\tikzset{
    basic/.style  = {draw, text width=2cm, drop shadow, font=\sffamily, rectangle},
    root/.style   = {basic, rounded corners=2pt, thin, align=center,
                     fill=green!30},
    onode/.style = {basic, thin, align=center, fill=green!60,text width=3cm,},
    tnode/.style = {basic, thin, align=left, fill=pink!60, text width=6.5em},
    edge from parent/.style={->, >={latex}, draw=black, edge from parent fork right}
}

\newcommand{\cmark}{\ding{51}}%
\newcommand{\xmark}{\ding{55}}%

\newtheorem{post}{Postulat}
\newtheorem{mydef}{Définition}

\begin{document}
\title{Résumé Journalier}
\author{Joffrey Hérard}
\date{13 avril 2017} 

\maketitle
\section{Test local avec des machines virtuelles KVM/QEMU}
\begin{minted}{bash} 
#Sur l'hote
egrep '^flags.*(vmx|svm)' /proc/cpuinfo
#Si on a pas un rendu vide on peut virtualiser

apt-get install kvm qemu-kvm
qemu-img create hda.deb.raw 2G
#ou on peut faire aussi 
kvm-img create hda.deb.raw 2G
kvm hda.deb.raw -m 512 -cdrom ../deb/debian.iso -boot d
kvm hda.deb.raw -m 512 -vnc :<PORT> -daemonize 

#Controle distant 
sudo apt install xtightvncviewer
xtightvncviewer <ADRESSE-MACHINE>:PORT
#Par contre pour la gestion massive on va utiliser un ensemble d'outils les deux installer sur les machines hote et distante(Controle)
sudo apt-get install libvirt-bin virtinst

sudo usermod -a -G libvirtd user
sudo usermod -a -G kvm user

virsh -c qemu:///system list 
#Non vide = tableau afficher vide == OK 
sudo apt-get install virt-manager
\end{minted} 

\newpage
\section{Installation de la machine Debian}

\subsection{Configuration réseau}

\begin{lstlisting}[language=bash,caption={ifconfig},frame=single]
eth0      Link encap:Ethernet  HWaddr d0:67:e5:e9:09:4a  
          inet adr:10.22.9.17  Bcast:10.22.9.255  Masque:255.255.255.0
          adr inet6: 2001:660:4601:7008:d267:e5ff:fee9:94a/64 Scope:Global
          adr inet6: fe80::d267:e5ff:fee9:94a/64 Scope:Lien
          UP BROADCAST RUNNING MULTICAST  MTU:1500  Metric:1
          RX packets:29226 errors:0 dropped:77 overruns:0 frame:0
          TX packets:887 errors:0 dropped:0 overruns:0 carrier:0
          collisions:0 lg file transmission:1000 
          RX bytes:3177132 (3.0 MiB)  TX bytes:61561 (60.1 KiB)

eth1      Link encap:Ethernet  HWaddr d0:67:e5:e9:09:4c  
          inet adr:10.22.9.156  Bcast:10.22.9.255  Masque:255.255.255.0
          adr inet6: 2001:660:4601:7008:d267:e5ff:fee9:94c/64 Scope:Global
          adr inet6: fe80::d267:e5ff:fee9:94c/64 Scope:Lien
          UP BROADCAST RUNNING MULTICAST  MTU:1500  Metric:1
          RX packets:28908 errors:0 dropped:0 overruns:0 frame:0
          TX packets:5720 errors:0 dropped:0 overruns:0 carrier:0
          collisions:0 lg file transmission:1000 
          RX bytes:4942677 (4.7 MiB)  TX bytes:1061567 (1.0 MiB)

eth2      Link encap:Ethernet  HWaddr d0:67:e5:e9:09:4e  
          inet adr:10.22.9.157  Bcast:10.22.9.255  Masque:255.255.255.0
          adr inet6: 2001:660:4601:7008:d267:e5ff:fee9:94e/64 Scope:Global
          adr inet6: fe80::d267:e5ff:fee9:94e/64 Scope:Lien
          UP BROADCAST RUNNING MULTICAST  MTU:1500  Metric:1
          RX packets:30562 errors:0 dropped:0 overruns:0 frame:0
          TX packets:394 errors:0 dropped:0 overruns:0 carrier:0
          collisions:0 lg file transmission:1000 
          RX bytes:2710930 (2.5 MiB)  TX bytes:131434 (128.3 KiB)

eth3      Link encap:Ethernet  HWaddr d0:67:e5:e9:09:50  
          UP BROADCAST MULTICAST  MTU:1500  Metric:1
          RX packets:0 errors:0 dropped:0 overruns:0 frame:0
          TX packets:0 errors:0 dropped:0 overruns:0 carrier:0
          collisions:0 lg file transmission:1000 
          RX bytes:0 (0.0 B)  TX bytes:0 (0.0 B)

lo        Link encap:Boucle locale  
          inet adr:127.0.0.1  Masque:255.0.0.0
          adr inet6: ::1/128 Scope:Hôte
          UP LOOPBACK RUNNING  MTU:65536  Metric:1
          RX packets:38 errors:0 dropped:0 overruns:0 frame:0
          TX packets:38 errors:0 dropped:0 overruns:0 carrier:0
    
\end{lstlisting}

\subsection{/etc/network/interfaces}

\begin{lstlisting}[language=bash,caption={/etc/network/interfaces},frame=single]
# This file describes the network interfaces available on your system
# and how to activate them. For more information, see interfaces(5).

source /etc/network/interfaces.d/*

# The loopback network interface
auto lo
iface lo inet loopback

# The primary network interface
auto eth0
allow-hotplug eth0
iface eth0 inet static
	address 10.22.9.17
	netmask 255.255.255.0
	network 10.22.9.0
	broadcast 10.22.9.255
	gateway 10.22.9.254
	dns-nameservers 8.8.8.8
iface eth1 inet manual
iface eth2 inet manual
iface eth3 inet manual
# ANCIEN FICHIER 
# auto vmbr0
# iface vmbr0 inet static 
# 	adress 172.18.10.1
# 	netmask 255.255.255.0
# 	bridge_ports eth0.1020
# 	bridge_stp off
# 	bridge_fd 0

# auto vmbr1
# iface vmbr1 inet static 
# 	adress 10.22.9.17
# 	netmask 255.255.255
# 	gateway 10.22.9.254
# 	bridge_ports eth0
# 	bridge_stp off
# 	bridge_fd 0 
#######

\end{lstlisting}
Ce fichier est une copie de l'original, celui donner par la machine apres installation ce trouve dans le meme répertoire mais nommé interfaces-old . 
\newpage
\subsection{Espace disque}
\begin{lstlisting}[language=bash,caption={df},frame=single]
df -h
Sys. de fichiers Taille Utilisé Dispo Uti% Monté sur
/dev/md0p2         2,6T    1,2G  2,5T   1% /
udev                10M       0   10M   0% /dev
tmpfs               48G    8,8M   48G   1% /run
tmpfs              119G       0  119G   0% /dev/shm
tmpfs              5,0M       0  5,0M   0% /run/lock
tmpfs              119G       0  119G   0% /sys/fs/cgroup
\end{lstlisting}

\begin{lstlisting}[language=bash,caption={dmesg},frame=single]
dmesg | grep Memory:
[    0.000000] Memory: 248107144K/251645024K available (5247K kernel
 code, 947K rwdata, 1832K rodata, 1208K init, 840K bss, 3537880K
  reserved)
\end{lstlisting}
\subsection{Carte graphique }
Il n'y as pas de carte graphique "suffisante" pour effectuer des tests. Elle est extrêmement faible.
\newpage

\section{Lancement des premiers tests exemples}

Lancement sur 25 machines Debian lance par docker. des tests suivants (Resultat demain je l’espère )
\begin{itemize}
\item  7zip-compresssion  	v1.6.2 $\rightarrow$ Test sur la compression
\item  Crafty				v1.3.1 $\rightarrow$ Test IA sur le jeux des échecs
\item  TCSP					v1.2.1 $\rightarrow$ Test IA sur le jeux des échecs
\item  Povray				v1.1.3 $\rightarrow$ The Persistence of Vision Raytracer 
\item  Systester			v1.0.0 $\rightarrow$ Calcul de Pi sur 4 millions de digit 
\end{itemize}
\newpage
\begin{thebibliography}{9}


\end{thebibliography}
\end{document}