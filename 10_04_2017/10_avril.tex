\documentclass[french]{article}
\usepackage{amssymb, amsmath, mathtools} %pour les mathématiques
\usepackage{fontspec}
\usepackage{xunicode}
\usepackage[a4paper]{geometry}
\usepackage{babel}
\usepackage{hyperref}
\usepackage{pifont}
\usepackage{tikz}
\usepackage{listings}
\usepackage{minted}
\lstset{basicstyle=\ttfamily,
  showstringspaces=false,
  commentstyle=\color{red},
  keywordstyle=\color{blue}
}
\usetikzlibrary{arrows.meta,shapes,positioning,shadows,trees}

\tikzset{
    basic/.style  = {draw, text width=2cm, drop shadow, font=\sffamily, rectangle},
    root/.style   = {basic, rounded corners=2pt, thin, align=center,
                     fill=green!30},
    onode/.style = {basic, thin, align=center, fill=green!60,text width=3cm,},
    tnode/.style = {basic, thin, align=left, fill=pink!60, text width=6.5em},
    edge from parent/.style={->, >={latex}, draw=black, edge from parent fork right}
}

\newcommand{\cmark}{\ding{51}}%
\newcommand{\xmark}{\ding{55}}%

\newtheorem{post}{Postulat}
\newtheorem{mydef}{Définition}

\begin{document}
\title{Résumé Journalier}
\author{Joffrey Hérard}
\date{10 avril 2017} 

\maketitle
\section{Test local avec machines VirtualBox}
\subsection{Test local avec des conteneurs LXC}
\subsubsection{Configuration de base}
Virtualbox emule une debian 8 qui va simuler notre Hote(notre futur serveur ) et les invite seront des machines LXC.
\begin{minted}{bash}
#Preparation pour l'hote
echo 'deb http://httpredir.debian.org/debian jessie-backports main' >> /etc/apt/sources.list
echo 'deb http://httpredir.debian.org/debian stretch main' >> /etc/apt/sources.list
apt-get update
apt-get install python-zmq python-systemd/jessie-backports python-tornado/jessie-backports \
salt-common/stretch bridge-utils libvirt-bin debootstrap iproute2

apt-get install lxc lxctl
apt-get install salt-master salt-minion
lxc-checkconfig 

#Pour creer des machines virtuelles debian 8  
# repeter l'operations autant de fois que ce faire 
lxc-create -n deb102 -t debian -- -r jessie
lxc-create -n deb103 -t debian -- -r jessie


#Creation du brige : 
ip link add name br0 type bridge
ip link set eth0 down
ip addr flush dev eth0
ip link set eth0 up 
ip addr add 10.0.2.15/24 broadcast 10.0.2.255 dev br0
ip link set dev br0 up
ip link set eth0 master br0
bridge link 

#Configuration des machines Debians respective

# Template used to create this container: /usr/share/lxc/templates/lxc-debian
# Parameters passed to the template: -r jessie
# Template script checksum (SHA-1): 2ad4d9cfe8988ae453172bd4fe3b06cf91756168
# For additional config options, please look at lxc.container.conf(5)

# Uncomment the following line to support nesting containers:
#lxc.include = /usr/share/lxc/config/nesting.conf
# (Be aware this has security implications)


lxc.rootfs = /var/lib/lxc/deb103/rootfs
lxc.rootfs.backend = dir

# Common configuration
lxc.include = /usr/share/lxc/config/debian.common.conf

# Container specific configuration
lxc.tty = 4
lxc.utsname = deb103
lxc.arch = amd64
lxc.network.type = veth
lxc.network.flags = up
lxc.rootfs = /var/lib/lxc/deb10X/rootfs
lxc.network.link = br0
lxc.network.hwaddr = 00:FF:AA:00:00:01
lxc.network.ipv4 = 10.0.2.21/24
lxc.network.ipv4.gateway = 10.0.2.15

#Ne pas oublier de faire la modif du fichier
nano /etc/sysctl.conf
#uncomment ipv4 forward =1 

#Sur les conteneurs LXC 
lxc-attach -n deb102 apt-get install salt-minion iputils-ping wget
lxc-attach -n deb103 apt-get install salt-minion iputils-ping wget
lxc-attach -n deb102 apt-get -f install
lxc-attach -n deb103 apt-get -f install

lxc-attach -n deb102 apt-get  install php5-cli php5-gd
lxc-attach -n deb103 apt-get  install php5-cli php5-gd 
lxc-attach -n deb102 echo "master: 10.0.2.15" >> /etc/salt/minion
lxc-attach -n deb103 echo "master: 10.0.2.15" >> /etc/salt/minion
lxc-attach -n deb102 echo salt-minion start
lxc-attach -n deb103 echo salt-minion start
#Sur la debian Hote
salt-master start &
salt-key --accept-all

#Maintenant on peut faire ce que l'on veut par exemple 
salt '*' test.ping
deb103:
    True
debJo:
    True
deb102:
    True
salt '*' network.interfaces 
\end{minted}

\subsubsection{Rédaction des différents fichier Salt pour l'automatisation des taches}
\begin{minted}{bash}
salt '*' cmd.run 'wget http://phoronix-test-suite.com/releases/repo/pts.debian/files/phoronix-test-suite_7.0.1_all.deb' 
\end{minted}
\begin{minted}{bash}
salt '*' cmd.run ' dpkg -i phoronix-test-suite_7.0.1_all.deb' 
\end{minted}

Maintenant que on a installer Phoronix, on va exporter le fichier de configuration tel que les fichiers soit sauvegarde avec un prompt réduit.
\newpage
\begin{thebibliography}{9}

   \bibitem{Documentation de SaltStack}
       
          \url{https://docs.saltstack.com/en/latest/topics/installation/debian.html}.

\end{thebibliography}

\end{document}