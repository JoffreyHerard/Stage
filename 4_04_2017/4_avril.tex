\documentclass[french]{article}
\usepackage{amssymb, amsmath, mathtools} %pour les mathématiques
\usepackage{fontspec}
\usepackage{xunicode}
\usepackage[a4paper]{geometry}
\usepackage{babel}
\usepackage{hyperref}
\usepackage{pifont}

\newcommand{\cmark}{\ding{51}}%
\newcommand{\xmark}{\ding{55}}%

\newtheorem{post}{Postulat}
\begin{document}
\title{Résumé Journalier}
\author{Joffrey Hérard}
\date{\today} 

\maketitle
\section{Les VMs}
\paragraph{Liste de mes connaissances des différentes technologies de virtualisation}
\begin{itemize}
	\item VirtualBox
	\item VMWARE
	\item LXC
	\item Docker
	\item QEMU
	\item KVM
	\item HyperV
	\item Proxmox VE
	\item Oracle VM Server 
	\item Hyper-V Containers
\end{itemize}

Réorganisation des différentes VM par tableau :   
\begin{table}[!h]
\centering
\caption{Différents services de Virtualisations }
\begin{tabular}{|l|l|}
\hline
Conteneurs         & Hyperviseurs Types 1 et 2                                                                              \\ \hline
LXC                & Virtualbox                                                                                             \\
Docker             & VMWare                                                                                                 \\
Vagrant            & Qemu                                                                                                   \\
Hyper-V Containers & \begin{tabular}[c]{@{}l@{}}KVM\\ Proxmox(Basé sur KVM)\\ Oracle VM Server\\ Hyper-V\\\end{tabular} \\ \hline
\end{tabular}
\end{table}

\paragraph{Quoi garder?}
Choix sur les différentes VM :  
\begin{table}[!h]
\centering
\begin{tabular}{lll}
LXC   \checkmark \par           & Docker  \checkmark \par            &  HyperV \xmark \\
VMWare  \xmark         & VirtualBox   \checkmark \par       & Qemu \xmark   \\
Oracle VM Server \xmark & Proxmox  \cmark          & KVM \xmark    \\
Hyper-V Containers \xmark          &  &
\end{tabular}
\end{table}
\newpage
Justification des choix cas par cas :
\begin{post}
LXC : Gestionnaire de Conteneurs Linux, utilisé aussi souvent que Docker (NB : Docker basé sur LXC)
\end{post}


\begin{post}
Docker : Gestionnaire de Conteneurs, que j'ai utilisé très souvent. (NB : Docker basé sur LXC)
\end{post}

\begin{post}
VirtualBox : Utilisé de manière courante aussi, de plus c'est un Hyperviseur classique de Type 2.
\end{post}

\begin{post}
Proxmox VE : Utilisé d'un point de vue utilisateurs, mais il complète la liste en étant un Hyperviseur de type 1.
\end{post}

\begin{post}
VMware : Je ne sais pas si l'on a une licence à notre disposition ? 
\end{post}

\begin{post}
Qemu : Utilisé quelque fois cependant, basé sur KVM(Comme Proxmox VE) et XEN(Connu que de nom). 
\end{post}


\begin{post}
Oracle VM Server : Jamais abordé ..et je ne sais pas si l'on a une licence à notre disposition ? 
\end{post}

\begin{post}
KVM : HyperViseur de type 1. Proxmox étant basé sur KVM et étant lui aussi un Hyperviseur de Type 1, 
\end{post}

\begin{post}
Hyper-V Containers : Jamais abordé .. ne connaissait pas son existence avant de faire de la recherche sur le sujet .
\end{post}

\begin{post}
Hyper-V : Jamais abordé.
\end{post}


\section{Les tests}
\paragraph{Test de performance}
\paragraph{Test de montée en charge}
\paragraph{Test de dégradations}
\paragraph{Test de robustesse}
\paragraph{Test aux limites}
\paragraph{Benchmark}
\newpage
\begin{thebibliography}{9}

   \bibitem{Docker}
          Documentation Docker
          \url{https://docs.docker.com/}.
   \bibitem{HyperV Conteneurs}
          Hyper-V Containers
          \url{https://docs.microsoft.com/en-us/virtualization/windowscontainers}.
   \bibitem{Linux containers}
         Linux Containers
         \url{https://linuxcontainers.org/fr/}.
   \bibitem{Virtual Box}
         Virtual Box
         \url{ https://www.virtualbox.org/}.
   \bibitem{Proxmox VE}
         Proxmox VE
         \url{https://www.proxmox.com/en/}.
   \bibitem{VmWare}
         VmWare
         \url{ http://www.vmware.com/fr.html}.
   \bibitem{QEMU}
         QEMU
         \url{http://www.qemu-project.org/   }.
   \bibitem{Hyper-V}
         Hyper-V
         \url{https://www.microsoft.com/fr-fr/cloud-platform/server-virtualization}.      
   \bibitem{KVM}
         KVM
         \url{https://www.linux-kvm.org/}.

\end{thebibliography}


\end{document}