\documentclass[french]{article}
\usepackage{amssymb, amsmath, mathtools} %pour les mathématiques
\usepackage{fontspec}
\usepackage{xunicode}
\usepackage[a4paper]{geometry}
\usepackage{babel}
\usepackage{hyperref}
\usepackage{pifont}
\usepackage{tikz}
\usepackage{listings}
\usepackage{minted}
\lstset{basicstyle=\ttfamily,
  showstringspaces=false,
  commentstyle=\color{red},
  keywordstyle=\color{blue}
}
\usetikzlibrary{arrows.meta,shapes,positioning,shadows,trees}

\tikzset{
    basic/.style  = {draw, text width=2cm, drop shadow, font=\sffamily, rectangle},
    root/.style   = {basic, rounded corners=2pt, thin, align=center,
                     fill=green!30},
    onode/.style = {basic, thin, align=center, fill=green!60,text width=3cm,},
    tnode/.style = {basic, thin, align=left, fill=pink!60, text width=6.5em},
    edge from parent/.style={->, >={latex}, draw=black, edge from parent fork right}
}

\newcommand{\cmark}{\ding{51}}%
\newcommand{\xmark}{\ding{55}}%

\newtheorem{post}{Postulat}
\newtheorem{mydef}{Définition}

\begin{document}
\title{Résumé Journalier}
\author{Joffrey Hérard}
\date{21 avril 2017} 

\maketitle
\section{Mise en place machines virtuelles avec libvirt}

Ensemble de commande 
\begin{lstlisting}[language=bash,caption={}]
sudo apt-get install kvm qemu-kvm libvirt-bin virtinst
sudo usermod -a -G libvirtd user
sudo usermod -a -G kvm user
$ virsh list
$ virsh start foo

$ virsh reboot foo

$ virsh shutdown foo

$ virsh suspend foo

$ virsh resume foo
$ virsh dumpxml foo > /tmp/foo.xml

$ virsh connect qemu+ssh://jherard@194.57.105.124:6299/system
\end{lstlisting}
Les VM sont crée avec virt-manager
\begin{lstlisting}[language=bash,caption={}]
sudo apt-get install virt-manager
#VM configuration supplementaire : 
# /etc/default/grub
GRUB_CMDLINE_LINUX_DEFAULT="console=tty0 console=ttyS0"
GRUB_TERMINAL="serial console"
update-grub

\end{lstlisting}

\newpage
\begin{thebibliography}{9}

   \bibitem{Site de Libvirt}
       
          \url{https://libvirt.org/}.

\end{thebibliography}

\end{document}