\documentclass[french]{article}
\usepackage{amssymb, amsmath, mathtools} %pour les mathématiques
\usepackage{fontspec}
\usepackage{xunicode}
\usepackage[a4paper]{geometry}
\usepackage{babel}
\usepackage{hyperref}
\usepackage{pifont}
\usepackage{tikz}
\usetikzlibrary{arrows.meta,shapes,positioning,shadows,trees}

\tikzset{
    basic/.style  = {draw, text width=2cm, drop shadow, font=\sffamily, rectangle},
    root/.style   = {basic, rounded corners=2pt, thin, align=center,
                     fill=green!30},
    onode/.style = {basic, thin, align=center, fill=green!60,text width=3cm,},
    tnode/.style = {basic, thin, align=left, fill=pink!60, text width=6.5em},
    edge from parent/.style={->, >={latex}, draw=black, edge from parent fork right}
}

\newcommand{\cmark}{\ding{51}}%
\newcommand{\xmark}{\ding{55}}%

\newtheorem{post}{Postulat}
\newtheorem{mydef}{Définition}

\begin{document}
\title{Résumé Journalier}
\author{Joffrey Hérard}
\date{\today} 

\maketitle

\section{WBS}
\begin{tikzpicture}[
      every node/.style = {draw, rounded corners=3pt, semithick, drop shadow},
            ROOT/.style = {top color=green!60!blue, bottom color=blue!60!green,
                             inner sep=2mm, text=white, font=\bfseries},
              L1/.style = {fill=blue!20},
              L2/.style = {fill=orange!30},
              L3/.style = {fill=green!30, grow=down, xshift=1em, anchor=west, 
      edge from parent path={(\tikzparentnode.south) |- (\tikzchildnode.west)}},
edge from parent/.style = {draw, thick},
              LD/.style = {level distance=#1ex},
             LD1/.style = {level distance=6ex},
             LD2/.style = {level distance=12ex},
             LD3/.style = {level distance=18ex},
         level 1/.style = {sibling distance=32mm}
                        ]
    % Parents
\node[ROOT] {root}
    [edge from parent fork down]
    child{node[L2] {AAA AAA AAA}
      child[L3,LD1]   {node[L3]   {A1}}
      child[L3,LD2]  {node[L3]   {A2}}
      child[L3,LD3]   {node[L3]   {A3}}
            }
    child{node[L2] {BB  BB BB BB}
      child[L3,LD1]  {node[L3]   {B1}}
      child[L3,LD2]  {node[L3]   {B2}}
            }
    child {node[L2] {C CC CCC}
      child[L3,LD1] {node[L3]   {C1}}
      child[L3,LD2] {node[L3]   {C2 C2 C2}
          child[L3,LD1] {node[L3,fill=red!30]   {C1}}
          child[L3,LD2] {node[L3,fill=red!30]   {C2}} 
                        }     
      child[L3,LD=30]  {node[L3]   {C2 C2 C2}}
            };
\end{tikzpicture}
\section{Détails sur chaque test}
\subsection{Lien avec le Hardware}
A voir 
\subsection{Impact d'un Hyperviseur}
Sachant que le monde des Hyperviseurs est assez vaste, chacun à une politique différentes ne seras-ce que par son choix sur le type de son hyperviseur (1 et 2). Par conséquent il y a forcement des Hyperviseurs qui vont avoir une politique différente sur l'accès concurrent à des ressources. Les hyperviseurs ont besoin d'isoler les interruptions et les accès à la mémoire. C'est très coûteux en termes de performances. Les surcoûts en termes de performances pour virtualiser un système comportent trois aspects principaux : la virtualisation du processeur, de la mémoire et des entrées/sorties.

De plus, au delà de la manière de gérer, les limites imposée elle même par les hyperviseur/conteneurs. 
\newpage
\begin{thebibliography}{9}

   \bibitem{Docker}
          Performance des hyperviseurs.
          \url{https://fr.wikipedia.org/wiki/Performances_des_hyperviseurs}.
   \bibitem{Docker}
          An Analysis of Performance Interference Effects in Virtual Environments
          \url{http://ieeexplore.ieee.org/document/4211036/?arnumber=4211036}
               
\end{thebibliography}

\end{document}