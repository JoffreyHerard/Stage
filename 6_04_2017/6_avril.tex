\documentclass[french]{article}
\usepackage{amssymb, amsmath, mathtools} %pour les mathématiques
\usepackage{fontspec}
\usepackage{xunicode}
\usepackage[a4paper]{geometry}
\usepackage{babel}
\usepackage{hyperref}
\usepackage{pifont}
\usepackage{tikz}
\usetikzlibrary{arrows.meta,shapes,positioning,shadows,trees}

\tikzset{
    basic/.style  = {draw, text width=2cm, drop shadow, font=\sffamily, rectangle},
    root/.style   = {basic, rounded corners=2pt, thin, align=center,
                     fill=green!30},
    onode/.style = {basic, thin, align=center, fill=green!60,text width=3cm,},
    tnode/.style = {basic, thin, align=left, fill=pink!60, text width=6.5em},
    edge from parent/.style={->, >={latex}, draw=black, edge from parent fork right}
}

\newcommand{\cmark}{\ding{51}}%
\newcommand{\xmark}{\ding{55}}%

\newtheorem{post}{Postulat}
\newtheorem{mydef}{Définition}

\begin{document}
\title{Résumé Journalier}
\author{Joffrey Hérard}
\date{\today} 

\maketitle

\section{WBS}

Ebauche standard de base a modifier . 
\begin{tikzpicture}[
      every node/.style = {draw, rounded corners=3pt, semithick, drop shadow},
            ROOT/.style = {top color=green!60!blue, bottom color=blue!60!green,
                             inner sep=2mm, text=white, font=\bfseries},
              L1/.style = {fill=blue!20},
              L2/.style = {fill=orange!30},
              L3/.style = {fill=green!30, grow=down, xshift=1em, anchor=west, 
      edge from parent path={(\tikzparentnode.south) |- (\tikzchildnode.west)}},
edge from parent/.style = {draw, thick},
              LD/.style = {level distance=#1ex},
             LD1/.style = {level distance=6ex},
             LD2/.style = {level distance=12ex},
             LD3/.style = {level distance=18ex},
         level 1/.style = {sibling distance=32mm}
                        ]
    % Parents
\node[ROOT] {root}
    [edge from parent fork down]
    child{node[L2] {AAA AAA AAA}
      child[L3,LD1]   {node[L3]   {A1}}
      child[L3,LD2]  {node[L3]   {A2}}
      child[L3,LD3]   {node[L3]   {A3}}
            }
    child{node[L2] {BB  BB BB BB}
      child[L3,LD1]  {node[L3]   {B1}}
      child[L3,LD2]  {node[L3]   {B2}}
            }
    child {node[L2] {C CC CCC}
      child[L3,LD1] {node[L3]   {C1}}
      child[L3,LD2] {node[L3]   {C2 C2 C2}
          child[L3,LD1] {node[L3,fill=red!30]   {C1}}
          child[L3,LD2] {node[L3,fill=red!30]   {C2}} 
                        }     
      child[L3,LD=30]  {node[L3]   {C2 C2 C2}}
            };
\end{tikzpicture}
\section{Détails sur chaque test}
\subsection{Lien avec le Hardware}
Au delà de la manière de gérer de chaque technologies de conteneurisation ou d'hypervision , il y a les limites imposée elle même par les hyperviseur/conteneurs. 
\begin{table}[]
\centering
\caption{My caption}
\label{my-label}
\begin{tabular}{|l|l|l|l|}
\hline
Systeme & Ressources                           & Windows Server 2012 r2 & VMWare                                                 \\ \hline
Hote    & Processeurs logiques                 & 320                    & 2 CPU physiques (480 CPU logiques), sans limitation du \\ \hline
        & Memoire physiques                    & 4 To                   & 4 To (12 To sur systèmes certifiés)                    \\ \hline
        & Processeurs virtuel par hote         & 2048                   &                                                     to do    \\ \hline
Invite  & Socket virtuel par machine virtuelle & 64                     &                                                       to do   \\ \hline
        & Memoire par machine virtuelle        & 1 To                   & Illimité                                               \\ \hline
        & Machine virtuelle active par hote    & 1024                   & 1024 VM, 4096 vCPU                                     \\ \hline
Cluster & Nombre de noeud maximum              & 64                     & 8                                                      \\ \hline
        & Nombre max de machine virtuelle      & 1000                   &                                                       to do   \\ \hline
\end{tabular}
\end{table}

\subsection{Impact d'un Hyperviseur}
Sachant que le monde des Hyperviseurs est assez vaste, chacun à une politique différentes ne seras-ce que par son choix sur le type de son hyperviseur (1 et 2). Par conséquent il y a forcement des Hyperviseurs qui vont avoir une politique différente sur l'accès concurrent à des ressources. Les hyperviseurs ont besoin d'isoler les interruptions et les accès à la mémoire. C'est très coûteux en termes de performances. Les surcoûts en termes de performances pour virtualiser un système comportent trois aspects principaux : la virtualisation du processeur, de la mémoire et des entrées/sorties.

\subparagraph{Processeur}
L'utilisation d'un hyperviseur au-dessous du système d'exploitation diffère du schéma habituel où le système d'exploitation est l'entité la plus privilégiée dans le système.De nombreuses architectures de processeur ne fournissent que deux niveaux de privilèges. Dans une virtualisation efficace.L'utilisation du processeur a des implications critiques sur les performances des autres caractéristiques du système.Dans l'évaluation des performances d'une machine virtuelle, les processus sont gérés par la machine virtuelle à la place du système d'exploitation sous-jacent. Les threads émulés des environnements multi-thread, en dehors des capacités du système d'exploitation d'origine, sont gérés dans l'espace utilisateur à la place de l'espace noyau, permettant le travail avec des environnements sans support natif des threads. De bonnes performances sur un microprocesseur multi-cœur sont obtenues grâce à l'implémentation de threads natifs pouvant attribuer automatiquement le travail à plusieurs processeurs, ils permettent un démarrage plus rapide de processus sur certaines machines virtuelles.

\subsection{Impact d'un conteneur}
A la base, le concept de conteneurisation permet aux instances virtuelles de partager un système d'exploitation hôte unique, avec ses fichiers binaires, bibliothèques ou pilotes. 
Cette approche réduit le gaspillage des ressources car chaque conteneur ne renferme que l'application et les fichiers binaires ou bibliothèques associés. On utilise donc le même système d'exploitation (OS) hôte pour plusieurs conteneurs, au lieu d'installer un OS (et d'en acheter la licence) pour chaque VM invitée.Le conteneur de chaque application étant libéré de la charge d'un OS, il est nettement plus petit, plus facile à migrer ou à télécharger, plus rapide à sauvegarder ou à restaurer. Enfin, il exige moins de mémoire. La conteneurisation permet au serveur d'héberger potentiellement beaucoup plus de conteneurs que s'il s'agissait de machines virtuelles. La différence en termes d'occupation peut être considérable, car un serveur donné accueillera de 10 à 100 fois plus d'instances de conteneur que d'instances d'application sur VM.


\newpage
\begin{thebibliography}{9}

   \bibitem{Docker}
          Performance des hyperviseurs.
          \url{https://fr.wikipedia.org/wiki/Performances_des_hyperviseurs}.
   \bibitem{Art1}
          An Analysis of Performance Interference Effects in Virtual Environments
          \url{http://ieeexplore.ieee.org/document/4211036/?arnumber=4211036}
   \bibitem{Art2}
          The impact of Docker containers on the performance of genomic pipelines
          \url {https://www.ncbi.nlm.nih.gov/pmc/articles/PMC4586803/}
      
      
      
\end{thebibliography}

\end{document}